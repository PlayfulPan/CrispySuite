\documentclass[12pt, ThmSectionNumbering]{CrispyNotes}
\usepackage{CrispyMacros}

\fancyhead[L]{\thepage\hspace{0.5cm}\keyterm{Vector Spaces, Matrices, and Index Notation}}


\DeclareMathOperator{\Ind}{Ind}
\DeclareMathOperator{\Res}{Res}

\hypersetup{
                pdfpagemode=UseOutlines,
}

\addbibresource{Bibliography.bib}
\begin{document}
\hypertarget{bm:Introduction}{\section*{Introduction}}
\markright{Introduction}
\bookmark[dest=bm:Introduction, level=section]{Introduction}

In this chapter we begin our study in earnest. The first order of business is to build up enough machinery to give a proper definition of manifolds. The chief problem with the provisional definition given in Chapter 1 is that it depends on having an \enquote{ambient Euclidean space} in which our $n$-manifold lives. This introduces a great deal of extraneous structure that is irrelevant to our purposes. Instead, we would like to view a manifold as a mathematical object in its own right, not as a subset of some larger space. The key concept that makes this possible is that of a topological space, which is the main topic of this chapter.

\section{Topological Spaces}


In this chapter we begin our study in earnest. The first order of business is to build up enough machinery to give a proper definition of manifolds. The chief problem with the provisional definition given in Chapter 1 is that it depends on having an \enquote{ambient Euclidean space} in which our $n$-manifold lives. This introduces a great deal of extraneous structure that is irrelevant to our purposes. Instead, we would like to view a manifold as a mathematical object in its own right, not as a subset of some larger space. The key concept that makes this possible is that of a topological space, which is the main topic of this chapter.

\subsection{Local Compactness and Paracompactness}

The next topological property of manifolds that we need is local compactness (see Appendix A for the definition).
\begin{proposition}[Manifolds Are Locally Compact]\label{prop1}
    Every topological manifold is locally compact.
\end{proposition}
\begin{proof}
    See \cite[47]{leeIntroductionSmoothManifolds2013}.
\end{proof}

Another key topological property possessed by manifolds is called \emph{paracompactness}. It is a consequence of local compactness and second-countability, and in fact is one of the main reasons why second-countability is included in the definition of manifolds.

Let $M$ be a topological space. A collection $\mathcal{X}$ of subsets of $M$ is said to be \keyterm{locally finite} if each point of $M$ has a neighborhood that intersects at most finitely many of the sets in $X$. Given a cover $\mathcal{U}$ of $M$, another cover $\mathcal{V}$ is called a \keyterm{refinement of $\symbfcal{U}$} if for each $V \in \mathcal{V}$ there exists some $U \in \mathcal{U}$ such that $V \subseteq U$. We say that $M$ is \keyterm{paracompact} if every open cover of $M$ admits an open, locally finite refinement.

\begin{theorem}[The Residue Theorem]
    Suppose $f$ is a meromorphic function in $\Omega$. Let $A$ be the set of points in $\Omega$ at which $f$ has poles $\xdif{Q}$. If $\Gamma$ is a cycle in $\Omega\smallsetminus A$ such that
        \begin{equation}\label{eq1}
            \Ind_\Gamma (\alpha) = 0 \quad \text{for all} \quad \alpha\notin\Omega\mathc
        \end{equation}
    then
        \begin{equation}\label{eq12}
            \frac{1}{2\pi i}\int_{\Gamma} f(z)\odif{z} = \sum_{a\in A} \Res(f; a)\, \Ind_\Gamma (a)\mathp
        \end{equation}
\end{theorem}
\begin{proof}
    Let $B = \Set{a \in A \given \Ind_{\Gamma}(a)\neq 0}$. Let $W$ be the complement of $\Gamma^*$. Then $\Ind_{\Gamma}(z)$ is constant in each component of $V$ in $W$. If $V$ is unbounded, or if $V$ intersects $\setcomp{\Omega}$, \subcref{eq1} implies that $\Ind_{\Gamma}(z)=0$ for every $z\in V$. Since $A$ has no limit point in $\Omega$, we conclude that $B$ is a finite set.

    The sum in \subcref{eq12}, though formally infinite, is therefore actually finite.

    Let $a_1, \ldots, a_n$ be the points of $B$, let $Q_1, \ldots, Q_n$ be the principal parts of $f$ at $a_1, \ldots, a_n$, and put $g=f-(Q_1+\dots+Q_n)$.\footnote{If $B=\emptyset$, a possibility which is not excluded, then $g=f$.} Put $\Omega_0= \Omega\smallsetminus(A\smallsetminus B)$. Since $g$ has singularities at $a_1, \ldots, a_n$, \Cref{prop1}, applied to the function $g$ and the open set $\Omega_0$, shows that \subcref{eq1}
        \begin{equation}\label{d}
            \int_{\Gamma} g(z) \odif{z} = 0\mathp
        \end{equation}
    Hence
        \begin{equation}\label{d2}
          \frac{1}{2\pi i}\int_{\Gamma} f(z)\odif{z} = \sum_{k=1}^{n}\frac{1}{2\pi i}\int_{\Gamma} Q_k(z) \odif{z} = \sum_{k=1}^{n}\Res(Q_k; a_k)\Ind_{\Gamma}(a_k)\mathc
        \end{equation}
    and since $f$ and $Q_k$ have the same residue at $a_k$, we obtain \subcref{eq12}.
\end{proof}

This proof may be found in \cite{rudinRealComplexAnalysis1986}. Observe that the crucial ingredient in the proof of the Riesz-Fischer theorem is the fact that $L^2$ is complete. This is so well recognized that the name \enquote{Riesz-Fisher theorem} is sometimes given to the theorem which asserts the completeness of $L^2$, or even of any $L^p$.

\begin{theorem}\label{theorem2}
    For a linear transformation $\Lambda$ of a normed linear space $X$ into a normed linear space $Y$, each of the following three conditions implies the other two:
    \begin{thmitems}
        \item\label{theorem2:a}
            $\Lambda$ is bounded.

        \item\label{theorem2:b}
            $\Lambda$ is continuous.

        \item\label{theorem2:c}
            $\Lambda$ is continuous at one point of $X$.
    \end{thmitems}
\end{theorem}
\begin{proof}[of {\subcref[theorem2]{theorem2:a}}]
    Since $\norm{\Lambda(x_1-x_2)} \leq \norm{\Lambda}\,\norm{x_1-x_2}$, it is clear that \subcref{theorem2:a} implies \subcref{theorem2:b}, and \subcref{theorem2:b} implies \subcref{theorem2:c} trivially. Suppose $\Lambda$ is continuous at $x_0$. To each $\epsilon>0$ one can then find a $\delta>0$ so that $\norm{x-x_0}<\delta$ implies $\norm{\Lambda x - \Lambda x_0}<\epsilon$. In other words, $\norm{x}<\delta$ implies
        \begin{equation*}
            \norm{\Lambda(x_0+x)-\Lambda x_0} <\epsilon\mathp
        \end{equation*}
    But then linearity of $\Lambda$ shows that $\norm{\Gamma x}<\epsilon$. Hence $\norm{\Lambda}\leq \epsilon\slash\delta$, and \subcref{theorem2:c}.
\end{proof}

\subsection{Elementary Properties of Measures}
\begin{definition}~
    \begin{thmitems}
        \item
            A \keyterm{positive measure} is a function $\mu$, defined on a $\sigma$-algebra $\mathfrak{M}$, whose range is in $[0, \infty]$ and which is \keyterm{countable additive}. This means that if $\Set{A_i}$ is a \emph{disjoint} countable collection of members of $\mathfrak{M}$, then
                \begin{equation}\label{d4}
                    \mu\biggl(\bigcup_{i=1}^\infty A_i\biggr) = \sum_{i=1}^\infty \mu(A_i)\mathp
                \end{equation}
            To avoid trivialities, we shall also assume that $\mu(A)<\infty$ for at least one $A\in\mathfrak{M}$.

        \item
            A \keyterm{measure space} is a measurable space which has a positive measure defined on the $\sigma$-algebra of its measurable sets.

        \item
            A \keyterm{complex measure} is a complex-valued countably additive function defined on a $\sigma$-algebra.
  \end{thmitems}
\end{definition}

Hey look its a quantum state! \subcref[theorem2]{theorem2:a}
    \begin{equation*}
      \ket{\phi}
    \end{equation*}

Here is a commutative diagram! \(\func{f}{X}{Y}\)
        \begin{equation*}
        \begin{tikzcd}
            T
            \arrow[drr, bend left, "x"]
            \arrow[ddr, bend right, "y"]
            \arrow[dr, dotted, "{(x,y)}" description] & & \\
                & X \times_Z Y \arrow[r, "p"] \arrow[d, "q"]
                & X \arrow[d, "f"] \\
                & Y \arrow[r, "g"]
                & Z
        \end{tikzcd}
        \end{equation*}

\pagebreak
\hypertarget{bm:References}{}
\bookmark[dest=bm:References, level=section]{References}
\printbibliography
\end{document} 